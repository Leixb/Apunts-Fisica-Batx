\documentclass[12pt,oneside]{article}
\usepackage[a4paper, left=2.5cm, right=2.5cm, top=2.5cm, bottom=1in]{geometry}

\usepackage{amsmath} 
\usepackage{amssymb} 
\usepackage{amstext}
\usepackage{amsthm}

% for bibliography:
\usepackage{comment} 
%\usepackage[ backend=biber, style=chicago ]{biblatex}
\usepackage[
backend=biber,
style=numeric,
]{biblatex}
\DeclareNameAlias{default}{last-first}

\addbibresource{biblio.bib}
% see:
% https://www.sharelatex.com/learn/Bibliography_management_in_LaTeX#The_bibliography_file

\usepackage[export]{adjustbox}
\usepackage{tikz}
\usetikzlibrary{arrows}
\usetikzlibrary{scopes}
\usetikzlibrary{babel}

% For cross references 
%\usepackage{hyperref} 
\usepackage[colorlinks = true]{hyperref}
\usepackage[catalan]{varioref}
%\usepackage{cleveref}
%hyperref configuration so that it doesn't contrast so much colorlinks,
\usepackage{xcolor} 
\hypersetup{ 
   linkcolor={black},
   citecolor={black}, 
   %linkcolor={red!50!black},
   %citecolor={blue!50!black}, 
   urlcolor={blue!80!black} }

% Custom Math operators (functions not in italic in math mode):
\DeclareMathOperator{\arcsec}{arcsec} 
\DeclareMathOperator{\arccot}{arccot}
\DeclareMathOperator{\arccsc}{arccsc} 
\DeclareMathOperator{\cis}{cis}

\usepackage[catalan]{babel} %Names in spanish
\usepackage[utf8]{inputenc} %Use unicode
\usepackage[T1]{fontenc}
\usepackage{csquotes} %For bibliography quotations
\DeclareQuoteAlias{spanish}{catalan}

\usepackage{array} 
\usepackage{float}  %Force tables and images position (H and H!)
\usepackage{wrapfig} %Wrap images like in HTML
\usepackage{listings} %For code blocks
\usepackage{color}  %Custom colors for syntax highlight in listings

\usepackage{tabularx,colortbl, booktabs} %Better tables
\usepackage[alsoload=hep]{siunitx} %Better tables and SI units and uncertainties
\usepackage{longtable}
\sisetup{separate-uncertainty=true}
\sisetup{locale = FR} %commas and so on for spanish
\sisetup{
  per-mode=fraction,
  fraction-function=\nicefrac
}
\usepackage{multirow}
\usepackage{multicol}
\usepackage{makecell}%Slit cell in lines and more formating options inside table

\usepackage{datetime} %To customize date

\newdateformat{monthyeardate}{%
    \monthname[\THEMONTH], \THEYEAR}
%Now \monthyeardate\today gives the date without the day

\usepackage[framemethod=tikz]{mdframed} 
\usepackage{nicefrac} %nice fractions in one line

%Subfigures
\usepackage{subcaption} 
\usepackage{relsize} %Bigger math with mathlarger{___}

\usepackage[bottom]{footmisc} %footnote at the bottom

%\usepackage{multicol}

\definecolor{codegreen}{rgb}{0,0.6,0} 
\definecolor{codegray}{rgb}{0.5,0.5,0.5}
\definecolor{codepurple}{rgb}{0.58,0,0.82}
\definecolor{backcolour}{rgb}{0.95,0.95,0.92}

\lstdefinestyle{mystyle}{ backgroundcolor=\color{backcolour},
    commentstyle=\color{codegreen}, keywordstyle=\color{blue},
    numberstyle=\tiny\color{codegray}, stringstyle=\color{red},
    identifierstyle=\color{black}, basicstyle=\footnotesize,
    %breakatwhitespace=false,         
    breaklines=true,                 
    %captionpos=b,                    keepspaces=true,                 
    numbers=left,                    numbersep=5pt,
    showspaces=false,                
    %showstringspaces=false, showtabs=false,                  
    tabsize=4 }

\lstset{style=mystyle}


%\renewcommand{\figurename}{Fig.} \renewcommand{\tablename}{Tabla}
%tabla-es in babel better

\definecolor{lightblue}{RGB}{135,206,250}

% Add command before appendix session for page numbering: A-1
\newcommand{\appendixpagenumbering}{
    \break
    \pagenumbering{arabic}
    \renewcommand{\thepage}{\thesection-\arabic{page}}
}

\newcommand{\whitepage}{
    \clearpage\thispagestyle{empty}\addtocounter{page}{-1} \newpage \clearpage
}



\usetikzlibrary{decorations.pathmorphing,patterns}
\usetikzlibrary{3d}
\usepackage[siunitx]{circuitikz}

\let\oldsection\section
\renewcommand\section{\clearpage\oldsection}

\newcommand{\ihat}{\hat{\imath}}
\newcommand{\jhat}{\hat{\jmath}}
\newcommand{\khat}{\hat{k}}

\begin{document}

\title {Apunts Física NS}
\author {Aleix Boné}
\date {}

\maketitle

\begin{center}
    \begin{circuitikz}
        \draw (0,0) to[battery,l=$V_1$] (0,4) to[resistor,l=$R_1$] (4,4)
        to[battery,l=$V_2$] (4,2) to[resistor,l=$R_2$] (4,0) to[resistor,l=$R_3$] (0,0);
        \draw (4,0) to (8,0) to[resistor,l=$R_4$] (8,4) to (4,4);
        \draw [<-,line width=2pt] (2,2) ++(138:5mm) --++(60:-1pt) arc (-220:40:5mm);
        \draw [->,line width=2pt] (6,2) ++(138:5mm) --++(60:-1pt) arc (-220:40:5mm);
        \draw[-latex] (-0.25,3) -> node[above,sloped] {$I$} (-0.25,4);

        \draw[fill] (4,4) circle (0.05);
        \draw[fill] (4,0) circle (0.05);

    \end{circuitikz}
\end{center}


\pagebreak

\tableofcontents
\pagebreak

\section{Cinemàtica}
\label{sec:cinematica}

\begin{align}
    \begin{aligned}[c]
        \vec{V_m}&=\frac{\Delta \vec{r}}{\Delta t} \\
        \vec{v}&=\frac{d \vec{r}}{d t}
    \end{aligned}
    \qquad
    \qquad
    \begin{aligned}[c]
        \vec{A_m}&=\frac{\Delta \vec{v}}{\Delta t}\\
        \vec{a}&=\frac{d \vec{v}}{d t} = \frac{d^2 \vec{r}}{d t^2}
    \end{aligned}
\end{align}

\subsection{Moviment Rectilini Uniforme (MRU)}
\begin{align}\label{equ:Vel. MRU}
    v &= ctt \notag \\
    v &= \frac{d \vec{r}}{d t} &\rightarrow \int_{r_0}^r dr &= \int_{t_0}^t v * dt \notag \\
    && v \int_0^t dt &= r \bigg |_{r_0}^r \notag\\
    r &= r_0 + v*t 
\end{align}

\subsection{Moviment Rectilini Uniformement Accelerat (MRUA)}
\begin{align}\label{equ:Vel. MRUA}
    a &= ctt \notag \\
    a &= \frac{d \vec{v}}{d t} &\rightarrow \int_{v_0}^v dv &= \int_{t_0}^t a * dt \notag \\
    && a \int_0^t dt &= v \bigg |_{v_0}^v \notag\\
    v &= v_0 + a*t 
\end{align}

\begin{align}\label{equ:Pos. MRUA}
    v &= v_0 + a*t  \notag \\
    v &= \frac{d \vec{r}}{d t} &\rightarrow \int_{r_0}^r dr &= \int_{t_0}^t v_0 + a*t * dt \notag \\
    && v_0 \int_0^t dt + \int_0^t a*t*dt &= r \bigg |_{r_0}^r \notag\\
    r - r_0 &= v_0*t+\frac{1}{2}at^2 
\end{align}

\subsection{Tir Parabòlic}
\begin{align}\label{Tir Parabolic}
    \vec{r} &= X(t)i & + Y(t)j \notag \\
    & MRU &  MRUA \notag \\
    & \downarrow  & \downarrow \notag \\
    \vec{r} &= (x_0+v_0*\cos(\theta)) \ihat & + (y_0 + v_0*\sin(\theta)*t - \frac{1}{2}gt^2) \jhat
\end{align}

Inclinació en un punt P:
\begin{equation}\label{eqi:Inclinacio tir parabolic}
    \beta = tan^{-1}\bigg(\bigg|\frac{V_Py}{V_Px}\bigg|\bigg)
\end{equation}

\subsection{Moviment Circular}
\begin{align*}
    R &= \text{radi} &= ctt \\
    \phi &= \text{angle} \\
    \omega &= \text{velocitat angular} &= \frac{\Delta \phi}{\Delta t} \\
    \alpha &= \text{acceleració angular} &= \frac{\Delta \omega}{\Delta t} \\
    T &= \text{període} &= \frac{2\pi}{\omega} \\
    v &= \omega * R \\
    \vec{a} &= \vec{a_c} + \vec{a_t} \\
    \vec{a_t} &= \alpha * R \\
    \vec{a_c} &= \omega^2*R &= \frac{v^2}{R}
\end{align*}

\section{Dinàmica}
\label{sec:dinamica}

\subsection{Lleis de Newton}
\subsubsection{1a llei (inèrcia)}
\begin{center}
    Si $\sum \vec{F} = 0 \Rightarrow$ objecte en MRU o repòs.
\end{center}

\subsubsection{2a llei (fonamental)}
\begin{align}
    \sum \vec{F}\ &\alpha\ \vec{a} \notag \\
    \downarrow & \notag \\
    \vec{F} &= m * \vec{a} \\ 
    \notag \\
    m &= \text{massa} \notag \\
    \text{(resistència }&\text{al moviment) }\notag \\
\end{align}

\subsubsection{3a llei (acció-reacció)}
\begin{align*}
    \vec{F}_{a \rightarrow b} &= -\vec{F}_{b \rightarrow a} \\
    |F_{a \rightarrow b}| &= |F_{b \rightarrow a}|
\end{align*}

\subsection{Força de fricció}

\begin{equation}
    \vec{F}_r = \mu * N
\end{equation}
Hi ha dos tipus de coeficients de fricció ($\mu$):
\begin{itemize}
    \item $\mu _e = $ coeficient de fricció estàtic (quan el cos està en repòs)
    \item $\mu _d = $ coeficient de fricció dinàmic (quan el cos està en moviment)
\end{itemize}
La fricció sempre va en sentit contrari al moviment.


\subsection{La màquina d'Atwood}
Una politja d'on hi pengen dues masses. Si $m_1 > m_2$:
\begin{equation}
    a = g\frac{m_1-m_2}{m_1+m_2}
\end{equation} 

\section{Treball i energia}

\begin{align}
    W &= \vec{F}*\Delta \vec{r} = F * \Delta r * cos\theta \\
    W &= \int \vec{F}dx \\
    W &= \Delta E_c \text{ (teorema treball-energia)} \\
    W &= \tau \theta
\end{align}

\subsection{Forces Conservatives}
$\vec{F}$ és conservativa si:
\begin{align*}
    \oint \vec{F}d\vec{r} &= 0 \\
    &\text{ó} \\
    \int \vec{F}d\vec{r} &= ctt
\end{align*}

\begin{center}
    (es conservativa si el treball d'un punt a un altre es constant independentment del camí)
\end{center}

\begin{align}
    W_{NC} &= \Delta E_M \\
    E_M &= E_P + E_c
\end{align}

\subsection{Conservació de l'energia}
Si l'una força $\vec{F}$ es conservativa, té associada una energia potencial $U$:
\begin{equation}
    U = - \int \vec{F} dx 
\end{equation}
\subsection{Potència}
\begin{equation}
    P = \frac{\Delta E}{\Delta t} \qquad \left[  \si{\watt} =\si{\joule\per\second} = \si{\kilo\gram\metre\squared\per\second\cubed}\right]
\end{equation}

\section{Xocs}
Hi ha dos tipus de xocs, els elàstics i els inelàstics:

\subsection{Impuls}
\begin{equation}
    \vec{I} = \vec{F}\Delta t = m \vec{a} \Delta t = m \Delta \vec{v} = \Delta \vec{p}
\end{equation}

\begin{equation}
    \frac{d\vec{p}}{dt} = \vec{F}
\end{equation}

\subsection{Inelàstics}
Son xocs irreals en el que es conserva l'energia:
\begin{align}
    \Delta E &= 0 \\
    \Delta \vec{p} &= 0
\end{align}



\subsection{Elàstics}
Son xocs en els que no es conserva l'energia degut a la pèrdua d'energia en forma de calor, soroll, deformació del material...
\begin{align}
    \Delta E &\neq 0 \qquad \rightarrow E_f = E_i - E_\text{perduda}\\
    \Delta \vec{p} &= 0
\end{align}

\subsection{Coeficient de restitució}
Modela la pèrdua d'energia en un xoc inelàstic:
\begin{equation}
    E_\text{perduda} = \frac{v_{f1}-v_{f2}}{v_{i1}-v_{i2}}
\end{equation}

\section{Dinàmica de rotació}
\subsection{Moment}
Si tenim una força $\vec{F}$ que actua sobre un cos a una distància del centre de masses definida pel vector $\vec{r}$, el moment que genera aquesta força es:
\begin{equation}
    \vec{\tau} = \vec{r} \times \vec{F}
\end{equation}
On $\times$ es el producte vectorial i per tant:
\begin{equation}
    \vec{\tau} \perp \vec{r}\ \&\ \vec{F}
\end{equation}

\begin{center}
    \begin{tikzpicture}
        \draw (0,0) circle (2);
        \draw[fill=white] (1.4142,1.4142) circle (0.2);
        \draw[fill=black] (1.4142,1.4142) circle (0.05);
        \draw[-latex,thick] (0,0) -- node[above] {$\vec{r}$} (1.27279220614,1.27279220614);
        \draw[-latex,thick] (1.4142,1.6142) -- node[left] {$\vec{F}$} (1.4142,3);
        \draw[dashed,gray!75] (1.55563491861,1.55563491861) -- (3,3);

        \draw[-latex] [domain=45:90] plot ({1*cos(\x)+1.4142}, {1*sin(\x)+1.4142}) ;
        \node[above] at (1.9,2.4) {$\theta$};
        \node[above] at (1.85,1.15) {$\vec{\tau}$};
    \end{tikzpicture}
\end{center}


Utilitzarem el símbol $\odot$ per denotar que $\vec{\tau}$ va \emph{cap a fora} i el símbol $\times$ per denotar que va \emph{cap a dins}
\begin{align}
    |\vec{\tau}| &= |\vec{r}||\vec{F}|\sin \theta \\
    &= m\vec{a}\vec{r} = m r^2 \alpha \quad \text{si $\theta = \pi$}\\
    |\vec{\tau}_\text{total}| &= \alpha \sum m_i r_i^2 = I\alpha
\end{align}

\paragraph{Moment d'inèrcia}
\begin{align}
    I &\equiv \sum m_i r_i^2 \\
    I &= \int r^2 dm
\end{align}
Per a un cilindre massís:
\begin{equation}
    I = \frac{1}{2}mr^2
\end{equation}

\subsection{Energia de rotació}
\begin{equation}
    E_r = \frac{1}{2}\omega^2I
\end{equation}
En general, l'energia cinètica es:
\begin{equation}
    E_c = E_{c1} + E_r = \frac{1}{2} m v^2 + \frac{1}{2}I\omega^2
\end{equation}

\subsection{Rodament amb desplaçament}
Si una esfera roda es compleix que:
\begin{equation}
    v_{cm} = \omega R
\end{equation}
Si $v_{cm} > \omega R$, l'esfera llisca.

\subsection{Moment angular(L)}
\begin{equation}
    \vec{L} = \vec{r} \times \vec{p}
\end{equation}

\begin{align}
    \vec{L} &= I \vec{\omega} \\
    \frac{d\vec{L}}{dt} &= \vec{\tau}
\end{align}

\subsubsection{Teorema de Steiner}
El moment d'inèrcia d'un eix desplaçat $d$ unitats del centre de masses es:
\begin{equation}
    I_d = I_{cm} + m d^2
\end{equation}

\section{Camp gravitatori}
\label{sec:camp_gravitatori}

\subsection{Llei de la gravitació universal de Newton}
\label{sub:llei_de_la_gravitacio_universal}
\begin{equation} \label{eq:gravitacio_universal}
    \vec{F}_{AB} = -G\frac{m_Am_B}{r^2} \hat{u}_r
\end{equation}

\begin{equation}\label{eq:constant_grav_universal}
    G = \SI{6.67408(31)e-11}{\newton\metre\squared\per\kilo\gram\squared}
\end{equation}

\begin{figure}[H]
    \centering
    \caption{Força gravitatòria entre dos cosos}
    \label{fig:forca_grav}
    \begin{tikzpicture}
        \draw [gray!75,thick, dashed] (-2,0) -- (10,0);
        \node [gray!75,left] at (-2,0) {$\hat{u}_r$};
        \draw [thick,fill=white] (0,0) circle (1) node {$m_A$};
        \draw [thick,fill=white] (7,0) circle (2) node {$m_B$};
        \draw [->,very thick] (1,0) -- node[above]{$F_{BA}$}(2.5,0);
        \draw [->,very thick] (5,0) -- node[above]{$F_{AB}$}(3.5,0);
        \draw [|-|,thick] (0,-2.5) -- node[above] {$r$} (7,-2.5);
    \end{tikzpicture}
\end{figure}

\vspace{1cm}

\begin{multicols}{2}

    \subsection{Intensitat del camp}
    \label{sub:intensitat_del_camp}

    \begin{equation}
        \vec{g} = -G \frac{M}{r^2} \hat{u}_r
    \end{equation}

    \subsection{Velocitat orbital}
    \label{sub:velocitat_orbital}

    \begin{equation}
        V = \sqrt{\frac{GM}{r}}
    \end{equation}

    \subsection{Velocitat d'escapament}
    \label{sub:velocitat_d_escapament}

    \begin{equation}
        \sqrt{2G\frac{M}{r}} = \sqrt{2}V_\text{orbital}
    \end{equation}

\end{multicols}

\subsection{Energies}
\label{sub:energies}

\begin{multicols}{2}
    \subsubsection{Energia Potencial}
    \label{ssub:energia_potencial}
    \begin{equation}
        E_p = -G \frac{Mm}{r}
    \end{equation}

    \subsubsection{Energia cinètica}
    \label{ssub:energia_cinetica}
    \begin{equation}
        E_c = \frac{1}{2} G\frac{Mm}{r}
    \end{equation}

    \subsubsection{Energia mecànica}
    \label{ssub:energia_mecanica}
    \begin{equation}
        E_m = E_c + E_p = -\frac{1}{2} G \frac{Mm}{r}
    \end{equation}
\end{multicols}

\pagebreak

\subsection{Lleis de Kepler}
\label{sub:lleis_de_kepler}

\begin{enumerate}
    \item Els planetes orbiten el sol de forma el·líptica i el sol ocupa un focus.
        \begin{equation}
            \varepsilon = \frac{a}{b} = \text{excentricitat de l'òrbita}
        \end{equation}
    \item El vector $\vec{r}$ escombra àrees iguals en temps iguals, per tant, la velocitat varia durant l'òrbita del planeta. 
    \item El període al quadrat es directament proporcional al radi de l'òrbita al cub. Suposant que l'\textbf{òrbita sigui circular}:
        \vspace{1cm}
        \begin{equation}
            T^2 = \frac{4\pi}{GM}r^3
        \end{equation}
\end{enumerate}

\vspace{1cm}

\begin{center}
    \begin{tikzpicture}
        \begin{scope}
            \clip (0,0) ellipse (5 and 3); 
            \fill[blue!15] (0,0) -- (1,3) -- (5,3) -- (5,0) -- (0,0);  
        \end{scope}
        \begin{scope}
            \clip (0,0) ellipse (5 and 3); 
            \fill[green!15] (0,0) -- (-3,-2) -- (-5,-3) -- (2,-3) -- (0,0);
        \end{scope}
        \node at (2,1) {$A$};
        \node at (-0.75,-2) {$A$};
        \shade[ball color=white] (0,0) circle (1) node {$M$};
        \draw (0,0) ellipse (5 and 3);
        %\draw (0,0) ellipse (5.2 and 3.2);
        \shade[ball color=white] (5,0) circle (0.3) node {$m$};
    \end{tikzpicture}
\end{center}

\section{Camp elèctric}
\label{sec:camp_electric}

\subsection{Llei de Coulomb}
\begin{equation}
    \vec{F} = k\frac{q_1q_2}{r^2} \hat{u} _r
\end{equation}
\begin{equation}
    k = \SI{9e9}{\newton\metre\squared\per\coulomb\squared} \qquad \text{(constant de coulomb)}
\end{equation}

\begin{figure}[H]
    \centering
    \caption{Força elèctrica entre dues càrregues}
    \label{fig:forca_ele}
    \begin{tikzpicture}
        \draw [gray!75,thick, dashed] (-3,0) -- (8,0);
        \node [gray!75,left] at (-3,0) {$\hat{u}_r$};
        \draw [gray!75,thick, dashed] (-3,-1.5) -- (8,-1.5);
        \node [gray!75,left] at (-3,-1.5) {$\hat{u}_r$};
        \draw [gray!75,thick, dashed] (-3,-3) -- (8,-3);
        \node [gray!75,left] at (-3,-3) {$\hat{u}_r$};

        \draw [thick,fill=blue!75] (0,0) circle (0.3) node[font=\bf] {$-$};
        \draw [thick,fill=red!75] (5,0) circle (0.3) node[font=\bf] {+};
        \draw [->,very thick] (0.3,0) -- node[above]{$F_{21}$}(2,0);
        \draw [->,very thick] (4.7,0) -- node[above]{$F_{12}$}(3,0);

        \draw [thick,fill=red!75] (0,-1.5) circle (0.3) node[font=\bf] {$+$};
        \draw [thick,fill=red!75] (5,-1.5) circle (0.3) node[font=\bf] {+};
        \draw [->,very thick] (-0.3,-1.5) -- node[above]{$F_{21}$}(-1.5,-1.5);
        \draw [->,very thick] (5.3,-1.5) -- node[above]{$F_{12}$}(7,-1.5);

        \draw [thick,fill=blue!75] (0,-3) circle (0.3) node[font=\bf] {$-$};
        \draw [thick,fill=blue!75] (5,-3) circle (0.3) node[font=\bf] {$-$};
        \draw [->,very thick] (-0.3,-3) -- node[above]{$F_{21}$}(-2,-3);
        \draw [->,very thick] (5.3,-3) -- node[above]{$F_{12}$}(7,-3);

        \draw [|-|,thick] (0,-3.7) -- node[above] {$r$} (5,-3.7);

        \node[above] at (0,0.3) {$q_1$};
        \node[above] at (5,0.3) {$q_2$};
        \node[above] at (0,0.3-1.5) {$q_1$};
        \node[above] at (5,0.3-1.5) {$q_2$};
        \node[above] at (0,0.3-3) {$q_1$};
        \node[above] at (5,0.3-3) {$q_2$};
    \end{tikzpicture}
\end{figure}

\subsection{Camp elèctric d'una càrrega}
\label{sub:camp_electric_duna_carrega}

\begin{multicols}{2}


    El camp elèctric $\vec{E}$ d'una càrrega $Q$ a una distància $r$ de $Q$ ve determinat per:
    \begin{equation}
        \vec{E} = k\frac{Q}{r^2} \hat{u}_r
    \end{equation}

    \columnbreak

    En general, el camp elèctric a un punt $P$ ve determinat per la suma vectorial
    dels camps elèctrics de totes les càrregues:
    \begin{equation}
        \vec{E}_P = \sum_{i=1}^n \vec{E}_i 
    \end{equation}


\end{multicols}



\subsection{Energia electrostàtica}
\begin{multicols}{2}

    \label{sub:energia_electrostatica}
    L'energia electrostàtica d'un sistema amb dues càrregues $q$ i $Q$ separades $r$ és:
    \begin{equation}
        U = k \frac{qQ}{r}
    \end{equation}

    En general, l'energia electrostàtica d'un sistema amb $n$ càrregues és la suma
    de l'energia electrostàtica entre totes les càrregues:

    \begin{equation}
        U_T = \mathlarger{\sum_{i=1}^n} \sum_{j=1}^n U_{ji}
    \end{equation}

    El treball es igual a menys la variació d'energia electrostàtica:

    \begin{equation}
        W = - \Delta U
    \end{equation}

\end{multicols}

\subsection{Potencial electrostàtic (Voltatge)}
\label{sub:potencial_electrostatic_voltatge_}

\begin{multicols}{2}

    \begin{equation}
        V = \frac{U}{q} = k \frac{Q}{r} \qquad \left[\si{\volt} = \si{\kilo\gram\metre\squared\per\second\cubed\per\ampere} \right]
    \end{equation}

    \subsubsection{Relació entre E i V}
    \label{ssub:relaci_entre_e_i_v}

    \begin{equation}
        \vec{E} = - \vec{\nabla} V = - \frac{dV}{dr} \hat{u}_r
    \end{equation}


\end{multicols}




\section{Camp magnètic}
\label{sec:camp_magnetic}

\begin{equation}
    \vec{B} \qquad \left[ \si{\tesla} = \si{\kilo\gram\per\ampere\per\second\squared}\right]
\end{equation}

\subsection{Llei de Lorenz}
\label{sub:llei_de_lorenz}

La força electromaètica induïda sobre una càrrega $q$ que travessa un camp magnètic $\vec{B}$ a una velocitat $\vec{v}$ ve determinada per:

\begin{equation}
    \vec{F} = q\vec{v}\times \vec{B}
\end{equation}

\begin{figure}[H]
    \centering
    \caption{Forca magnètica induïda sobre una càrrega en moviment}
    \label{fig:fem_induida}
    \begin{tikzpicture}
        %\draw[help lines] (0,0) grid (10,10);
        \node [gray!75,left] at (0,5) {$\vec{B}\quad$};
        \foreach \x in {0,...,5} {
            \foreach \y in {0,...,5}
            {
                \draw [color=gray!75,fill] (\x,\y) circle (0.05);
                \draw [color=gray!75,thick] (\x,\y) circle (0.15);
            }
        }
        \draw[<->,very thick] (1,0.5) -- node[right] {$\vec{F}_m$} (1,4) -- node[above] {$\vec{v}$} (4.5,4);
        \draw[->,dashed] (1,4) to[out=0,in=125] (5,2);
        \draw [very thick,fill=white] (1,4) circle (0.5) node {$+q$};

        \draw [fill] (0,0) circle (0.05);
        \draw [thick] (0,0) circle (0.15);
        \node[below left] (0,0) at (0,0) {$\khat$};
        \draw[<->,thick] (0.5,0) -- node[below right] {$\jhat$} (0,0) -- node[above left] {$\ihat$} (0,0.5);

    \end{tikzpicture}
\end{figure}

La força de Lorenz és sempre perpendicular a $\vec{v}$ i a $\vec{B}$, la magnitud de la força és 
\begin{equation}
    qvB\sin\widehat{vB}
\end{equation}

\subsection{Llei de Biot i Savart}
\label{sub:llei_de_biot_i_savart}

La llei de Biot i Savart relaciona el camp magnètic amb les corrents que
el creen.

\begin{equation}
    d\vec{B} = \frac{\mu_0I}{4\pi} \frac{d\vec{\ell}\times\vec{u}_r}{r^3}
\end{equation}
\begin{equation}
    \mu_0 = \SI{4\pi e-7}{\newton\per\ampere\squared} \qquad \text{(Permeabilitat elèctrica del buit)}
\end{equation}


\begin{center}
    \begin{tikzpicture}[cross/.style={path picture={ 
                \draw[black]
                (path picture bounding box.south east) -- (path picture bounding box.north west) (path picture bounding box.south west) -- (path picture bounding box.north east);
            }}]
        \draw[thick] (0,0) -- (0,4);
        \draw[very thick,|-|] (0,2) -- node[above,sloped] {$d\vec{\ell}$}(0,2.5);
        %\draw[dashed] (0,2.25) -- (3,2.25);
        \draw[thick,->] (0,2.25) -- node[above] {$r\vec{u}_r$} (3,2.25);
        \draw[->] (-0.25,0.5) -- node[above,sloped] {$I$} (-0.25,1);
        \draw[cross] (3+0.15,2.25) circle (0.15);
        \node[above] at (3+0.15,2.25+0.15) {$d\vec{B}$};
    \end{tikzpicture}
\end{center}

\subsection{Camp magnètic d'un cable recte i infinit}
\label{sub:camp_magn_tic_d_un_cable_recte_i_infinit}

\begin{equation}
    B = \frac{\mu_0I}{2\pi r}
\end{equation}
\begin{center}
    \begin{tikzpicture}[cross/.style={path picture={ 
                \draw[black]
                (path picture bounding box.south east) -- (path picture bounding box.north west) (path picture bounding box.south west) -- (path picture bounding box.north east);
            }}]
        \draw[gray!75,dashed] (0,2.25) ellipse(3 and 0.5);
        \begin{scope}
            \clip (-3,2.25) circle (0.15);
            \fill[white] (-3.5,2.25) rectangle (3.5,3.25);  
        \end{scope}
        \begin{scope}
            \clip (3,2.25) circle (0.15);
            \fill[white] (-3.5,2.25) rectangle (3.5,3.25);  
        \end{scope}
        \draw[very thick] (0,0) -- (0,4);
        \draw[dashed,thick] (0,2.25) -- node[above] {$r$} (3,2.25);
        \draw[dashed,thick] (0,2.25) -- node[above] {$r$} (-3,2.25);
        \draw[->] (-0.25,2-1) -- node[above,sloped] {$I$} (-0.25,2.5-1);
        \draw[cross] (3,2.25) circle (0.15);
        \draw (-3,2.25) circle (0.15);
        \draw[fill] (-3,2.25) circle (0.05);
        \node[above] at (3,2.25+0.15) {$\vec{B}$};
        \node[above] at (-3,2.25+0.15) {$\vec{B}$};
        \draw (0.15,2.5) -- (0.15,2.4) -- (0.25,2.4);
    \end{tikzpicture}
\end{center}


\subsection{Camp magnètic d'una espira}
\label{sub:camp_magnetic_d_una_espira}

El camp magnètic al centre d'una espira de radi $r$ on circula una intensitat
$I$ és:

\begin{equation}
    B = \frac{\mu_0I}{2R}
\end{equation}


\begin{center}
    \begin{tikzpicture}[cross/.style={path picture={ 
                \draw[black]
                (path picture bounding box.south east) -- (path picture bounding box.north west) (path picture bounding box.south west) -- (path picture bounding box.north east);
            }}]
        \draw[very thick] (0,0) circle (2);
        \draw node [above] at (0,2.25) {$I$};
        \draw[-latex] [domain=80:100] plot ({2.25*cos(\x)}, {2.25*sin(\x)});

        \draw (0,0) circle (0.25);
        \draw[fill] (0,0) circle (0.05);
        \draw node [above] at (0,0.25) {$\vec{B}$};

        \draw[dashed] (0,0) -- node[below] {$R$} (2,0);

    \end{tikzpicture}
\end{center}

El camp magnètic a una distància $z$ del centre de l'anella a través del eix és:

\begin{equation}
    B = \frac{\mu_0R^2I}{2\left(z^2 + R^2 \right)^\frac{3}{2}}
\end{equation}

\begin{center}
    \begin{tikzpicture}[cross/.style={path picture={ 
                \draw[black]
                (path picture bounding box.south east) -- (path picture bounding box.north west) (path picture bounding box.south west) -- (path picture bounding box.north east);
            }}]
        \draw[very thick] (0,0) circle (2);
        \draw node [above] at (0,2.25) {$I$};
        \draw[-latex] [domain=80:100] plot ({2.25*cos(\x)}, {2.25*sin(\x)});

        \draw (0,0) circle (0.25);
        \draw[fill] (0,0) circle (0.05);
        \draw node [above] at (0,0.25) {$\vec{B}$};

        \draw[dashed] (0,0) -- node[below] {$R$} (2,0);

        \draw[|-|,dashed] (3.5,-2) -- node[sloped,above] {$R$} (3.5,0);

        \draw[very thick] (4,-2) -- (4,2);

        \draw (4,2) circle (0.25);
        \draw[fill] (4,2) circle (0.05);
        \draw node [above] at (4,2.25) {$I$};

        \draw[cross] (4,-2) circle (0.25);

        \draw[fill] (6,0) circle (0.05);

        \draw[dashed] (4,0) -- node[below] {$z$} (6,0);
        \draw[-latex,very thick] (6,0) -- node[above] {$\vec{B}$} (8,0);

    \end{tikzpicture}
\end{center}

\subsection{Camp magnètic d'un solenoide}
\label{sub:camp_magnetic_d_un_solenoide}

\begin{equation}
    B = \mu_0nI
\end{equation}
\begin{equation}
    n = \frac{N}{\ell}
\end{equation}

On $N$ és el nombre d'espires i $\ell$ es la longitud del solenoide.

\begin{center}
    \begin{tikzpicture}[cross/.style={path picture={ 
                \draw[black]
                (path picture bounding box.south east) -- (path picture bounding box.north west) (path picture bounding box.south west) -- (path picture bounding box.north east);
            }}]
        \draw[very thick, decoration={aspect=0.3, segment length=20, amplitude=40,coil},decorate] (0,0) -- (5,0); 
        %\draw[very thick, decoration={aspect=0.3, segment length=20, amplitude=40,coil},decorate] (0,0) circle (3); 
        \draw[very thick] (0,-2) -- (0,0);
        \draw[-latex] (-0.25,-1.75) -- node[sloped,above] {$I$} (-0.25,-1.25);
        \draw[|-|] (0.25,-2) -- node[below] {$\ell$} (4.75,-2);

        \draw[-latex,very thick] (2,0) -- node[above] {$\vec{B}$} (4,0);
        \draw[fill] (2,0) circle (0.05);

    \end{tikzpicture}
\end{center}

\subsection{Camp magnètic d'un toroide}
\label{sub:camp_magnetic_d_un_toroide}

El camp magnètic al \textbf{interior} d'un toroide de radi $R$ on circula una
intensitat $I$ i amb $N$ espires es:

\begin{equation}
    B = \mu_0nI
\end{equation}
\begin{equation}
    n = \frac{N}{2\pi R} \qquad \to \qquad B = \frac{\mu_0NI}{2\pi}
\end{equation}

\vspace{1cm}

\begin{center}
    \begin{tikzpicture}[cross/.style={path picture={ 
                \draw[black]
                (path picture bounding box.south east) -- (path picture bounding box.north west) (path picture bounding box.south west) -- (path picture bounding box.north east);
            }}]
        \draw[very thick, decoration={aspect=0.3, segment length=20, amplitude=22,coil},decorate] (0,0) circle (3); 
        \draw[dashed,|-|] (0,0) -- node[above] {$R$}(3,0);
        \draw[line width=3pt,-latex] (-3,0) -- node[above,sloped] {$\vec{B}$} (-3,3);
        \draw[fill] (-3,0) circle (0.10);

        \draw[-latex] (3.3,0) -- node[above,sloped] {$I$} (3.4,-0.75);

    \end{tikzpicture}
\end{center}

\pagebreak

\subsection{Flux}
\label{sub:flux}

El flux magnètic $\Phi$ d'un cos amb un vector superfície $S$ on hi actua un
camp magnètic $B$ és:

\begin{equation}
    \Phi = \int \vec{B}  d\vec{S} \qquad \left[ \si{\weber} = \si{\joule\per\ampere} = \si{\volt\second} = \si{\kilo \gram \metre \squared \per \second \squared \per \ampere} \right]
\end{equation}

Si $\vec{B}$ i $\vec{S}$ són constants i no varia l'angle $\theta$ entre els
dos en cap punt de la superfície, podem simplificar l'expressió a:

\begin{equation}
    \Phi = BS\cos\theta
\end{equation}

\subsection{Llei de Faraday-Lenz (Inducció electromagnètica)}
\label{sub:llei_de_faraday_lenz_induccio_electromagnetica}

La força electromotriu induïda $\varepsilon$ és igual a la variació de flux $\Phi$ respecte el temps per el nombre d'espires.

\begin{equation}
    \varepsilon = -N \frac{d\Phi}{dt} \qquad \left[\si{\volt}\right]
\end{equation}

En un interval de temps discret:

\begin{equation}
    \varepsilon = -N \frac{\Delta \Phi}{\Delta t}
\end{equation}

\subsection{Llei de Gauss}
\label{sub:llei_de_gauss}

El flux en una \textbf{superfície tancada} $S$ és

\begin{equation}
    \Phi = \oint \vec{B} d\vec{S} = \frac{Q}{\varepsilon_0}
\end{equation}

\begin{equation}
    \varepsilon_0 = \SI{8.854187817\dots e-12}{\farad\per\metre} \qquad \text{(Permeabilitat elèctrica del buit)}
\end{equation}

\subsection{Transformadors (Inductància mútua)}
\label{sub:transformadors}

\begin{equation}
    \frac{\varepsilon_1}{\varepsilon_2} = \frac{N_1}{N_2}
\end{equation}

\begin{center}
    \begin{circuitikz}
        \draw
        (0,0) node[transformer] (T) {}
        node[ocirc] (A) at ([xshift=-1cm]T.A1) {}
        node[ocirc] (B) at ([xshift=-1cm]T.A2) {}
        node[ocirc] (C) at ([xshift=1cm]T.B1) {}
        node[ocirc] (D) at ([xshift=1cm]T.B2) {}
        %node[circ] (E) at ([xshift=0.4cm,yshift=-5pt]T.A1) {}
        %node[circ] (F) at ([xshift=-0.4cm,yshift=-5pt]T.B1) {}
        (T.A1) to[-o] (A)
        (T.A2) to [-o] (B) 
        (T.B1) to[-o] (C)
        (T.B2) to [-o] (D)
        (T.west) node{$L_1$}
        (T.east) node{$L_2$}
        ;
        ;
    \end{circuitikz}
\end{center}

\subsubsection{Autoinductància}
\label{ssub:autoinductancia}

\begin{equation}
    \varepsilon = L \frac{dI}{dt} 
\end{equation}

\begin{equation}
    L = \text{Coeficient d'inductància} \qquad \left[ \si{\henry} \right]
\end{equation}

\section{Moviment harmònic simple (MHS)}
\label{sec:moviment_harm_nic_simple}

Un cos segueix un moviment harmònic simple si es compleix que:
\begin{enumerate}
    \item La magnitud de la força (i com a conseqüent de l'acceleració)
        és proporcional al desplaçament respecte a un punt fix.
    \item La direcció de la força (i per tant de l'acceleració) és sempre en la
        direcció del punt fix.
\end{enumerate}
Per tant:
\begin{align}
    a &\propto -\Delta x\\
    a &= -k\Delta x
\end{align}

Les equacions de moviment per a un objecte que es mou seguint un moviment harmònic simple d'amplitud $A$, freqüència angular $\omega$ i fase $\phi$ amb un punt fix $x_0$són:

\begin{align}
    x(t) &= x_0 + A\sin(\phi_0+\omega t) \\
    v(t) &= A\omega\cos(\phi_0+\omega t) \\
    a(t) &= -A\omega^2\cos(\phi_0+\omega t)
\end{align}

El període $T$, el període angular $\omega$ i la freqüència $f$ segueixen la
relació:

\begin{align}
    T = \frac{1}{f} = \frac{2\pi}{\omega}
\end{align}

\subsection{Posició, velocitat i acceleració màximes}
\label{sub:posicio_velocitat_i_acceleracio_maximes}


\begin{align}
    x_\mathrm{màx} &= x_0 + A \\
    v_\mathrm{màx} &= A\omega \\
    a_\mathrm{màx} &= A\omega^2
\end{align}

\subsection{Llei de Hooke}
\label{sub:llei_de_hooke}

La força exercida per una molla amb una constant elàstica $k$ segueix un
moviment harmònic simple.

\begin{equation}
    F = -k\Delta x
\end{equation}

\begin{equation}
    E_p = \frac{1}{2}k\Delta x
\end{equation}

\subsection{Període d'un pèndul simple}
\label{sub:periode_d_un_pendul_simple}

El període $T$ d'un pèndul simple de longitud $\ell$ que oscil·la en graus petits \footnote{Inferiors a \SI{15}{\degree}} és:

\begin{equation}
    T=2\pi\sqrt{\frac{\ell}{g}}
\end{equation}




\section{Circuits elèctrics}
\label{sec:circuits_electrics}

\subsection{Resistències}
\label{sub:resistencies}

La resistència d'un material de longitud $\ell$ àrea $A$ i resistivitat
elèctrica $\rho$ \footnote{\emph{rho}} [\si{\ohm\metre}] és:

\begin{equation}
    R = \rho \frac{\ell}{A} \qquad \left[ \si{\ohm} = \si{\kilo\gram\metre\squared\per\second\cubed\per\ampere\squared} \right]
\end{equation}

\subsection{Intensitat}
\label{sub:intensitat}

\begin{equation}
    I = \frac{\Delta Q}{\Delta t} \qquad \left[\si{\ampere}\right]
\end{equation}

\subsection{Velocitat de deriva}
\label{sub:velocitat_de_deriva}
La velocitat $v$ a la que es mou una partícula de càrrega $Q$ a través d'un
conductor d'àrea $A$ \footnote{secció perpendicular al flux de càrregues} on
hi ha $n$ partícules per metre cúbic es relaciona amb la intensitat:

\begin{equation}
    I = nAvQ
\end{equation}

\subsection{Llei d'Ohm}
\label{sub:llei_d_ohm}

\begin{equation}
    V = IR
\end{equation}

\subsection{Efecte Joule}
\label{sub:efecte_joule}

La potència dissipada per l'efecte Joule ve determinada per:

\begin{equation}
    P = RI^2 = VI \qquad \left[ \si{\watt} =\si{\joule\per\second} = \si{\kilo\gram\metre\squared\per\second\cubed}\right]
\end{equation}

\begin{equation}
    Q = Pt \qquad \left[\si{\joule} = \si{\newton\metre} = \si{\kilo\gram\metre\squared\per\second\squared} \right]
\end{equation}


\subsubsection{En sèrie}
\label{ssub:en_serie}



\begin{center}
    \begin{circuitikz}
        \draw (0,0) to [resistor,l=$R_1$] (2.5,0) to [resistor,l=$R_2$] (5,0);
        \draw[dashed] (5,0) to  (6,0);
        \draw (6,0) to [resistor,l=$R_n$] (8.5,0);
        \node at (9,0) {$\equiv$};
        \draw (9.5,0) to [resistor,l=$R_T$] (12,0);
    \end{circuitikz}
\end{center}

\begin{equation}\label{eq:resist_series}
    R_T = \sum_{i=1}^n R_i \qquad I_T = I_i \quad \forall\; i\; \exists \{1,\dots,n\}
\end{equation}


\subsubsection{En paral·lel}
\label{ssub:en_paral_lel}

\begin{center}
    \begin{circuitikz}
        \draw (-0.5,0) -- (0,0);
        \draw (0,-0.5) to (0,2);
        \draw[dashed] (0,-0.5) to (0,-1.5);
        \draw[dashed] (5,-0.5) to (5,-1.5);
        \draw (0,-1.5) to (0,-2);
        \draw (5,-1.5) to (5,-2);
        \draw (5,-0.5) to (5,2);
        \draw (5.5,0) -- (5,0);

        \draw (0,2) to[resistor,l=$R_1$] (5,2);
        \draw (0,1) to[resistor,l=$R_2$] (5,1);
        \draw (0,0) to[resistor,l=$R_3$] (5,0);

        \draw (0,-2) to[resistor,l=$R_n$] (5,-2);

        \node at (6,0) {$\equiv$};

        \draw (7,0) to [resistor,l=$R_T$] (10,0);
        %\foreach \x in {0,...,5} {
    \end{circuitikz}
\end{center}

\begin{equation}\label{eq:resist_paralel}
    \frac{1}{R_T} = \sum_{i=1}^n \frac{1}{R_i} \qquad I_T = \sum_{i=1}^n I_i
\end{equation}

\subsection{Condensadors}
\label{sub:condensadors}

\begin{equation}
    C = \frac{Q}{\Delta V} \qquad \left[ \si{\coulomb\per\volt} = \si{\farad} \right]
\end{equation}

\subsubsection{Energia acumulada}
\label{ssub:energia_acumulada}

\begin{equation}
    U = \frac{1}{2}QV
\end{equation}



\subsubsection{En sèrie}
\label{ssub:en_serie}

\begin{equation}\label{eq:cond_serie}
    \frac{1}{C_T} = \sum_{i=1}^n \frac{1}{C_i}
\end{equation}

\begin{center}
    \begin{circuitikz}
        \draw (0,0) to [C,l=$C_1$] (2.5,0) to [C,l=$C_2$] (5,0);
        \draw[dashed] (5,0) to  (6,0);
        \draw (6,0) to [C,l=$C_n$] (8.5,0);
        \node at (9,0) {$\equiv$};
        \draw (9.5,0) to [C,l=$C_T$] (12,0);
    \end{circuitikz}
\end{center}

\subsubsection{En paral·lel}
\label{ssub:en_paral_lel}

\begin{equation}\label{eq:cond_paralel}
    C_T = \sum_{i=1}^n C_i
\end{equation}

\begin{center}
    \begin{circuitikz}
        \draw (-0.5,0) -- (0,0);
        \draw (0,-0.5) to (0,2);
        \draw[dashed] (0,-0.5) to (0,-1.5);
        \draw[dashed] (5,-0.5) to (5,-1.5);
        \draw (0,-1.5) to (0,-2);
        \draw (5,-1.5) to (5,-2);
        \draw (5,-0.5) to (5,2);
        \draw (5.5,0) -- (5,0);

        \draw (0,2) to[C,l=$C_1$] (5,2);
        \draw (0,0) to[C,l=$C_2$] (5,0);

        \draw (0,-2) to[C,l=$C_n$] (5,-2);

        \node at (6,0) {$\equiv$};

        \draw (7,0) to [C,l=$C_T$] (10,0);
        %\foreach \x in {0,...,5} {
    \end{circuitikz}
\end{center}

\subsection{Teorema de Kirchoff}
\label{sub:teorema_de_kirchoff}

\subsubsection{Nus}
\label{ssub:nus}

En un nus com el que es mostra a la figura \ref{fig:nus}, es compleix que\footnote{S'ha de tenir en compte el sentit tant de les intensitats com del voltatge}:

\begin{equation}
    \sum I = 0
\end{equation}

\begin{equation}
    \sum I_\text{entrants} = \sum I_\text{sortints}
\end{equation}

\begin{figure}[H]
    \caption{Nus} \label{fig:nus}
    \begin{center}
        \begin{circuitikz}
            \draw (-2,0) to (2,0);
            \draw (0,-2) to (0,2);
            \draw[fill] (0,0) circle (0.15);
            \draw[-latex] (-2,0.25) to node[above] {$I_{1}$} (-0.5,0.25);
            \draw[-latex] (2,-0.25) to node[below] {$I_{2}$} (0.5,-0.25);
            \draw[-latex] (0.25,2) to node[right] {$I_{3}$} (0.25,0.5);
            \draw[-latex] (-0.25,-2) to node[left] {$I_{4}$} (-0.25,-0.5);
        \end{circuitikz}
    \end{center}
\end{figure}

\subsubsection{Malles}
\label{ssub:malles}
En una malla com la que es mostra a la figura \ref{fig:malla}, es compleix que:
\begin{equation}
    \sum V_i = \sum I_iR_i
\end{equation}

\begin{figure}[H]
    \caption{Malla}\label{fig:malla}
    \begin{center}
        \begin{circuitikz}
            \draw (0,0) to[battery,l=$V_1$] (0,4) to[resistor,l=$R_1$] (4,4)
            to[battery,l=$V_2$] (4,2) to[resistor,l=$R_2$] (4,0) to[resistor,l=$R_3$] (0,0);
            \draw (4,0) to (8,0) to[resistor,l=$R_4$] (8,4) to (4,4);
            \draw [<-,line width=2pt] (2,2) ++(138:5mm) --++(60:-1pt) arc (-220:40:5mm);
            \draw [->,line width=2pt] (6,2) ++(138:5mm) --++(60:-1pt) arc (-220:40:5mm);
            \draw[-latex] (-0.25,3) -> node[above,sloped] {$I$} (-0.25,4);

            \draw[fill] (4,4) circle (0.05);
            \draw[fill] (4,0) circle (0.05);

        \end{circuitikz}
    \end{center}
\end{figure}






\section{Òptica}
\label{sec:optica}

\subsection{Índex de refracció entre dos medis}
\label{sub:index_de_refraccio_entre_dos_medis}

L'índex de refracció entre dos medis és igual a la velocitat de la llum en el
primer medi entre la velocitat de la llum en el segon medi:

\begin{equation}
    _1n_2 = \frac{v_1}{v_2}
\end{equation}

\subsection{Índex de refracció absolut}
\label{sub:index_de_refraccio_absolut}

L'índex de refracció absolut $n$ d'un medi és la relació entre la velocitat de
la llum en el buit $c$ i la velocitat de la llum en el medi $v$:

\begin{equation}
    n = \frac{c}{v}
\end{equation}

\begin{equation}
    c = \SI{299792458}{\metre \per \second} \approx \SI{3e9}{\metre\per\second}
\end{equation}

\subsection{Llei de Snell}
L'angle d'incidència $\ihat$ per l'índex de refracció absolut del primer medi
és igual a l'angle de refracció $\hat{r}$ per l'índex absolut del segon medi.
\begin{equation}
    n_1 \sin \ihat = n_2 \sin \hat{r}
\end{equation}

\begin{center}

    \begin{tikzpicture}
        \shade[top color = gray!50, bottom color = white] (-3,0) rectangle (3,-3);
        \draw[very thick] (-3,0) -- (3,0);
        \draw [domain=90:150] plot ({1*cos(\x)}, {1*sin(\x)});
        \draw [domain=90:30] plot ({1.5*cos(\x)}, {1.5*sin(\x)});
        \draw [domain=270:290] plot ({1.5*cos(\x)}, {1.5*sin(\x)});
        \draw[dashed] (0,3) -- (0,-3);
        \draw[very thick,-latex,blue] (-3,1.73205080757) -- (0,0);
        \draw[thick,-latex,blue!75] (0,0) -- (1.0919107028,-3);
        \draw[-latex,blue!25] (0,0) -- (3,1.73205080757);

        \node[above] at (-2.75,0) {$n_1$};
        \node[below] at (-2.75,0) {$n_2$};

        \node[above] at (-0.75,1) {$\ihat$};
        \node[above] at (0.9,1.35) {$\ihat$};
        \node[below] at (0.25,-1.45) {$\hat{r}$};
    \end{tikzpicture}
\end{center}

\subsection{lleis dels miralls}
\begin{equation}
    \frac{1}{s} + \frac{1}{s'} = \frac{2}{r}
\end{equation}

\subsection{lleis de la lent}
\begin{equation}
    \frac{1}{s} + \frac{1}{s'} = \frac{1}{f}
\end{equation}


\section{Fluids}
\label{sec:fluids}

\subsection{Principi de Pascal}
\label{sub:principi_de_pascal}

\begin{displayquote}
    La pressió exercida per un fluid incompressible en equilibri dins d'un
    recipient de parets rígides es transmet d'igual manera en totes direccions i en
    tots els seus punts.
\end{displayquote}

La diferència de pressió $\Delta P$ entre dos punts submergits a diferent
profunditat $h$ és igual a la diferència d'$h$ per la densitat del fluid $\rho$
per l'acceleració de la gravetat $g$:

\begin{equation}
    \Delta P = \rho g \Delta h \qquad \left[ \si{\pascal} = \si{\newton \per \metre \squared} = \si{\kilo \gram \per\metre \per\second\squared} \right]
\end{equation}

\begin{center}
    \begin{tikzpicture}
        \draw[blue,very thick,decorate,decoration={coil,aspect=0.1}] (-2,3.75) -- (1.75,3.75);
        \draw[very thick] (-2,4) -- (-2,0) -- (1.75,0) -- (1.75,4);
        \node[blue] at (-1.5,3) {$\rho$};
        \draw (-.25/2,3) circle (0.3) node {$B$};
        \draw (-0.25/2,1) circle (0.3) node {$A$};
        \draw[|-|] (0.5,3) -- node[right] {$\Delta h$} (0.5,1);
    \end{tikzpicture}
\end{center}

\subsection{Principi d'Arquímedes}
\label{sub:principi_d_arquimedes}

\begin{displayquote}
    Un cos insoluble totalment o parcialment submergit en un fluid (líquid o gas)
    en repòs rep una força de baix cap a dalt igual al pes del volum del fluid que
    desallotja.
\end{displayquote}

La força de flotació $\vec{B}$ d'un cos submergit un volum $V$ en un fuid de densitat $\rho$ i sobre el cual hi actua la força de la gravetat $\vec{g}$ és:

\begin{equation}
    \vec{B} = -\rho V \vec{g}
\end{equation}

\begin{center}
    \begin{tikzpicture}
        \draw[blue,very thick,decorate,decoration={coil,aspect=0.1}] (-2,3.75) -- (1.75,3.75);
        \draw[very thick] (-2,4) -- (-2,0) -- (1.75,0) -- (1.75,4);
        \shade[ball color = white] (0,2) circle (0.5) node {$V$};
        \draw[-latex] (0,2+0.5) -- node[sloped,above] {$\vec{B}$} (0,3);
        \draw[-latex] (0,2-0.5) -- node[sloped,above] {$\vec{g}$} (0,0.5);
        \node[blue] at (-1.5,3) {$\rho$};
    \end{tikzpicture}
\end{center}

\pagebreak

\subsection{Equació de continuïtat}
\label{sub:equacio_de_continuitat}

En un mateix fluid, es compleix que la velocitat del fluid en un punt per 
l'àrea de la secció perpendicular es igual a la velocitat en un altre punt
per l'àrea en aquell punt.

\begin{equation}
    A_i v_i = \text{ctt} 
\end{equation}

\begin{center}
    \begin{tikzpicture}
        \draw[fill=blue!25,draw=black] (0,0) ellipse (0.5 and 3);
        \draw[fill=red!25,draw=black] (5,0) ellipse (0.5*1.5/3 and 1.5);

        \node[blue!75] at (0,0) {$A_1$};
        \node[red!75] at (5,0) {$A_2$};

        \foreach \y in {2.5,2,...,-2.5} {
            \draw[-latex,blue] (-2,\y) -- (-0.75,\y);
            \draw[-latex,red,very thick] (5.5,\y*0.5) -- (7,\y*0.5);
            \draw[-latex,gray!75,thick] (0.75,\y) -- (4.5,\y*0.5);
        };
        %\foreach \y in {1.25,1,...,-1.25} {
        %\draw[-latex,red,very thick] (5.5,\y) -- (7,\y);
        %};

        \draw (0,3) -- (5,1.5);
        \draw (0,-3) -- (5,-1.5);
        \draw (5,1.5) -- (7,1.5);
        \draw (5,-1.5) -- (7,-1.5);
        \draw (0,3) -- (-2,3);
        \draw (0,-3) -- (-2,-3);
    \end{tikzpicture}
\end{center}

\subsection{Equació de Bernoulli}
\label{sub:equaci_de_bernoulli}

\begin{align}
    P_T &= P_s + P_d\\
    P_d &= \frac{1}{2}\rho v^2\\
    P_s &= \rho g h
\end{align}

\subsection{Nombre de Reynolds}
\label{sub:nombre_de_reynolds}

El nombre de Reynolds  Re caracteritza el moviment del fluid. En un tub, si
$\mathrm{Re} > 4000$ el règim és turbulent, si $\mathrm{Re} < 2300$ el règim és laminar.\\

El nombre de Reynolds Re per a un líquid de densitat $\rho$ i viscositat cinètica $\mu$ que circula a una velocitat $v$ per un tub de diàmetre $D_H$ és:

\begin{equation}
    \mathrm{Re} = {{\rho v D_H} \over {\mu}} 
\end{equation}

\section{Termodinàmica}
\label{sec:termodinamica}

\begin{align}
    PV &= nRT \\
    U &= \frac{3}{2}nRT \qquad \rightarrow \qquad U \propto T \\
    Q &= c_em\Delta T
\end{align}

\subsection{1r principi}
\label{sub:1r_principi}

La calor $Q\;[\si{\joule}]$ despresa durant la variació de les condicions d'un
gas és la suma del treball $W$ i la variació d'energia interna del gas durant
el procés:

\begin{align}
    Q &= W + \Delta U \\
    W &= \int P dV
\end{align}

\subsection{Isobàric}
\label{sub:isobaric}
El procés es produeix a \textbf{pressió} constant.

\begin{align}
    Q = P \Delta V + \Delta U
\end{align}

\subsection{Isotèrmic}
\label{sub:isotermic}
El procés es produeix a \textbf{temperatura} constant.

\begin{align}
    \Delta U &= 0\\
    Q = W &= nRT\ln\frac{V_B}{V_A}
\end{align}

\subsection{Isocor}
\label{sub:isocor}
El procés es produeix a \textbf{volum} constant.

\begin{align}
    W &= 0\\
    Q &= \Delta U
\end{align}

\pagebreak

\subsection{Adiabàtic}
\label{sub:adiabatic}
O isentròpic, el procés es produeix sense intercanvi de calor ($Q=0$).

\begin{align}
    Q &= 0\nonumber\\
    W &= -\Delta U \\
    P*V^\gamma &= ctt \\
    \frac{T}{V^{\gamma + 1}} &= ctt \\
    \gamma &= \frac{2}{3} \nonumber
\end{align}

\subsection{Eficiència}
\label{sub:eficiencia}

\begin{equation}
    \eta = \frac{W}{Q}
\end{equation}

\subsection{Segon principi}
\label{sub:segon_principi}

L'entropia $S$ d'un sistema es igual a la variació de calor $\Delta Q$ entre la
temperatura. L'entropia de l'univers sempre augmenta.

\begin{align}
    S &= \frac{\Delta Q}{T} \\
    S &= k_b \ln \Omega \\
    S_\text{univers} &> 0
\end{align}

$\Omega$ és el nombre de microestats possibles del sistema.

\begin{equation}
    k_b =  \SI{1.38064852(79)e-23}{\joule\per\kelvin} \qquad \text{(Constant de Boltzmann)}
\end{equation}

\section{Unitats del SI}
\label{sec:unitats_del_si}

\begin{table}[H]
    \centering
    \caption{Les 7 Unitats bàsiques del SI}
    \label{tab:si_unitats_basiques}
    \begin{tabular}{ccc}
        \toprule
        Símbol  &   Nom     &   Magnitud \\
        \midrule
        \si{\metre}     & metre     & Longitud \\
        \si{\kilo\gram} & kilogram  & Massa\\
        \si{\second}    & segon     & Temps \\
        \si{\ampere}    & ampere    & Intensitat elèctrica \\
        \si{\kelvin}    & Kelvin    & Temperatura \\
        \si{\mol}       & Mols      & Quantitat de substància\\
        \si{\candela}   & Candela   & Intensitat Lluminosa\\

        \bottomrule

    \end{tabular}
\end{table}

\begin{table}[H]
    \centering
    \caption{Prefixos del SI}
    \label{tab:prefixos_del_si}
    \begin{tabular}{ccc}
        \toprule
        Símbol  &   Nom     &   Factor \\
        \midrule
        \si{\deca}   & deca     &   $10^1$  \\
        \si{\hecto}  & hecto    &   $10^2$  \\
        \si{\kilo}   & quilo    &   $10^3$  \\
        \si{\mega}   & mega     &   $10^6$  \\
        \si{\giga}   & giga     &   $10^9$  \\
        \si{\tera}   & tera     &   $10^{12}$ \\
        \si{\peta}   & peta     &   $10^{15}$ \\
        \si{\exa}    & exa      &   $10^{18}$ \\
        \si{\zetta}  & zetta    &   $10^{21}$ \\
        \si{\yotta}  & yotta    &   $10^{24}$ \\
        \bottomrule
    \end{tabular}
    \qquad
    \begin{tabular}{ccc}
        \toprule
        Símbol  &   Nom     &   Factor \\
        \midrule
        \si{\deci}   & deci   &   $10^{-1}$  \\
        \si{\centi}  & centi  &   $10^{-2}$  \\
        \si{\milli}  & mil·li &   $10^{-3}$  \\
        \si{\micro}  & micro  &   $10^{-6}$  \\
        \si{\nano}   & nano   &   $10^{-9}$  \\
        \si{\pico}   & pico   &   $10^{-12}$ \\
        \si{\femto}  & femto  &   $10^{-15}$ \\
        \si{\atto}   & atto   &   $10^{-18}$ \\
        \si{\zepto}  & zepto  &   $10^{-21}$ \\
        \si{\yocto}  & yocto  &   $10^{-24}$ \\

        \bottomrule
    \end{tabular}
\end{table}


\begin{table}[H]
    \centering
    \caption{Unitats derivades}
    \label{tab:unidades_derivadas}
    \begin{tabular}{cccc}
        \toprule
        Símbol  &   Nom     & En funció d'altres &   En unitats bàsiques \\
        \midrule

        \si{\hertz} & hertz && \si{\per\second} \\
        \si{\newton} & newton && \si{\kilo\gram\metre\per\second\squared} \\
        \si{\pascal} & pascal & \si{\newton \per \metre \squared} & \si{\kilo \gram \per\metre \per\second\squared} \\
        \si{\joule} & joule & \si{\newton\metre} & \si{\kilo\gram\metre\squared\per\second\squared} \\
        \si{\watt} & watt &\si{\joule\per\second} & \si{\kilo\gram\metre\squared\per\second\cubed}\\
        \si{\coulomb} & coulomb && \si{\ampere\second}\\
        \si{\volt} & volt &\si{\ampere\ohm} = \si{\watt\per\ampere} = \si{\joule\per\coulomb}& \si{\kilo\gram\metre\squared\per\second\cubed\per\ampere} \\
        \si{\farad} & faraday & \si{\coulomb\per\volt} & \si{\second\tothe{4}\ampere\squared\per\kilo\gram\per\metre\squared} \\
        \si{\ohm} & ohm &\si{\volt\per\ampere} = \si{\per\siemens}& \si{\kilo\gram\metre\squared\per\second\cubed\per\ampere\squared} \\
        \si{\siemens} & siemens & \si{\ampere\per\volt} = \si{\per\ohm} & \si{\second\cubed\ampere\squared\per\kilo\gram\per\metre\squared} \\
        \si{\weber} & weber & \si{\joule\per\ampere} = \si{\volt\second} & \si{\kilo \gram \metre \squared \per \second \squared \per \ampere} \\
        \si{\tesla} & tesla &\si{\weber\per\metre\squared}& \si{\kilo\gram\per\ampere\per\second\squared}\\
        \si{\henry} & henry & \si{\weber\per\ampere} & \si{\kilo\gram\metre\squared\per\second\squared\per\ampere\squared}\\
        \bottomrule
    \end{tabular}
\end{table}

\end{document}
